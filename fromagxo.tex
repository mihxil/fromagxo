% HTML
\documentstyle[esperant,html]{article} %twocolum, a4wide
\title{Froma^go}
\author{Willem Elsschot}
\date{}

\def\cxapitro#1{\section{#1}}

\def\rrim#1{}
\def\rim#1{}
%\def\rim#1{\footnote{#1}}

%\def\a#1{(#1)}
%\def\ax#1#2{(#2)}

\def\a#1{}
\def\ax#1#2{}


\begin{document}
\maketitle
\begin{rawhtml}
<hr>
<ul>
<li> Elnederlandigita de <a href="mailto:michiel.meeuwissen@gmail.com">Michiel Meeuwissen</a>.</li>
<li> versio: $Id$</i>
<li><a href="./fromagxo-a5.pdf">fromagxo-a5.ps</a>:A5 70 pa^goj.</li>
<li><a href="./fromagxo-libreto.pdf">fromagxo-libreto.pdf</a>: A5 70 pa^goj). Por uzo kun dupa^ga printilo sur A4.</li>
<li><a href="./fromagxo-a4.pdf">fromagxo-a4.pdf</a>: A4</li>
<li><a href="./fromagxo.epub">fromagxo.epub</a>: e-libro</li>
<li><a href="../elsschot/esp.html">Informo pri Elsschot </a></li>
<li><a href="../esperanto.html"> Esperanta hejmpa^go </a></li>
<li> Mi numeris la alineojn por faciligi viajn komentojn. </li>
<li> Petu al mi se vi bezonas alian formon.</li>
</ul>
<h1>Enhavo</h1>
\end{rawhtml}
\input{fromagxox.tks}
\end{document}
